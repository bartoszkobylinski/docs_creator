\documentclass[11pt,a4paper]{article}
\usepackage[utf8]{inputenc}
\usepackage[T1]{fontenc}
\usepackage{geometry}
\usepackage{fancyhdr}
\usepackage{titlesec}
\usepackage{listings}
\usepackage{xcolor}
\usepackage{graphicx}
\usepackage{hyperref}
\usepackage{booktabs}
\usepackage{longtable}
\usepackage{amsmath}
\usepackage{amssymb}

% Page geometry
\geometry{
    a4paper,
    total={170mm,257mm},
    left=20mm,
    top=20mm,
}

% Colors
\definecolor{codegreen}{rgb}{0,0.6,0}
\definecolor{codegray}{rgb}{0.5,0.5,0.5}
\definecolor{codepurple}{rgb}{0.58,0,0.82}
\definecolor{backcolour}{rgb}{0.95,0.95,0.92}

% Code listing style
\lstdefinestyle{pythonstyle}{
    backgroundcolor=\color{backcolour},   
    commentstyle=\color{codegreen},
    keywordstyle=\color{magenta},
    numberstyle=\tiny\color{codegray},
    stringstyle=\color{codepurple},
    basicstyle=\ttfamily\footnotesize,
    breakatwhitespace=false,         
    breaklines=true,                 
    captionpos=b,                    
    keepspaces=true,                 
    numbers=left,                    
    numbersep=5pt,                  
    showspaces=false,                
    showstringspaces=false,
    showtabs=false,                  
    tabsize=2
}

\lstset{style=pythonstyle}

% Header and footer
\pagestyle{fancy}
\fancyhf{}
\rhead{\VAR{project_name}}
\lhead{Documentation}
\rfoot{Page \thepage}

% Hyperref setup
\hypersetup{
    colorlinks=true,
    linkcolor=blue,
    filecolor=magenta,      
    urlcolor=cyan,
    pdftitle={\VAR{project_name} Documentation},
    pdfauthor={Auto-generated},
}

% Title formatting
\titleformat{\section}
{\Large\bfseries\color{blue}}
{\thesection}
{1em}
{}

\titleformat{\subsection}
{\large\bfseries\color{blue!80}}
{\thesubsection}
{1em}
{}

\title{\VAR{project_name}\\Documentation}
\author{Auto-generated Documentation}
\date{\VAR{generation_date}}

\begin{document}

\maketitle
\thispagestyle{empty}

\newpage
\tableofcontents
\newpage

\section{Overview}

This document provides comprehensive documentation for \textbf{\VAR{project_name}}. It includes API documentation, code analysis, and coverage statistics.

\subsection{Documentation Statistics}

\begin{itemize}
    \item \textbf{Total Items:} \VAR{total_items}
    \item \textbf{Documented Items:} \VAR{documented_items}
    \item \textbf{Coverage:} \VAR{"%.1f"|format(coverage_percent)}\%
    \item \textbf{Modules:} \VAR{modules|length}
\end{itemize}

\subsection{Project Modules}

\begin{itemize}
\BLOCK{for module in modules}
    \item \texttt{\VAR{module}}
\BLOCK{endfor}
\end{itemize}

\section{Code Documentation}

This section contains detailed documentation for all code elements found in the project.

\BLOCK{for item_type, items in items_by_type.items()}
\subsection{\VAR{item_type.replace('_', ' ').title()}}

\BLOCK{for item in items}
\subsubsection{\texttt{\VAR{item.qualname|replace('_', '\\_')}}}

\begin{itemize}
    \item \textbf{Module:} \texttt{\VAR{item.module|replace('_', '\\_')}}
    \item \textbf{File:} \texttt{\VAR{item.file_path|replace('_', '\\_') if item.file_path else 'N/A'}}
    \item \textbf{Line:} \VAR{item.lineno if item.lineno else 'N/A'}
\end{itemize}

\BLOCK{if item.docstring}
\paragraph{Documentation:}
\begin{quote}
\VAR{item.docstring|replace('_', '\\_')|replace('#', '\\#')|replace('&', '\\&')|replace('%', '\\%')}
\end{quote}
\BLOCK{else}
\paragraph{Documentation:} \textit{No documentation available}
\BLOCK{endif}

\BLOCK{if item.first_lines}
\paragraph{Source Code Preview:}
\begin{lstlisting}[language=Python]
\VAR{item.first_lines}
\end{lstlisting}
\BLOCK{endif}

\vspace{1em}
\BLOCK{endfor}
\BLOCK{endfor}

\BLOCK{if uml_diagrams and uml_diagrams.get('main_diagram')}
\section{UML Diagrams}

This section contains UML diagrams that visualize the project structure and relationships.

\BLOCK{if uml_diagrams.analysis}
\subsection{Diagram Analysis}

\begin{itemize}
    \item \textbf{Classes Found:} \VAR{uml_diagrams.analysis.get('classes_found', 'N/A')}
    \item \textbf{Relationships Found:} \VAR{uml_diagrams.analysis.get('relationships_found', 'N/A')}
    \item \textbf{Packages:} \VAR{', '.join(uml_diagrams.analysis.get('packages', []))|replace('_', '\\_')}
\end{itemize}
\BLOCK{endif}

\subsection{Main Diagram}

The main UML diagram shows the overall structure of the system.

\paragraph{PlantUML Source:}
\BLOCK{if uml_diagrams.main_diagram.get('source')}
\begin{lstlisting}[language={}]
\VAR{uml_diagrams.main_diagram.source}
\end{lstlisting}
\BLOCK{endif}

\BLOCK{if uml_diagrams.get('additional_diagrams')}
\subsection{Additional Diagrams}

\BLOCK{for diagram_type, diagram_data in uml_diagrams.additional_diagrams.items()}
\subsubsection{\VAR{diagram_type.title()} Diagram}

\BLOCK{if diagram_data.get('source')}
\begin{lstlisting}[language={}]
\VAR{diagram_data.source}
\end{lstlisting}
\BLOCK{endif}
\BLOCK{endfor}
\BLOCK{endif}
\BLOCK{endif}

\section{Coverage Analysis}

\BLOCK{for item_type, items in items_by_type.items()}
\subsection{\VAR{item_type.replace('_', ' ').title()} Coverage}

\VAR{items|length} total items, \VAR{items|selectattr('docstring')|list|length} documented (\VAR{"%.1f"|format((items|selectattr('docstring')|list|length / items|length * 100) if items|length > 0 else 0)}\%).

\BLOCK{endfor}

\section{Conclusion}

This documentation was automatically generated from the codebase analysis. For the most up-to-date information, please refer to the source code and regenerate this documentation as needed.

\end{document}